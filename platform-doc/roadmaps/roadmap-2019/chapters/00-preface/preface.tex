\chapter{Preface}
\paragraph{Keywords:} \emph{Conceptual Modelling, Information Systems, Software Architecture, Software Engineering, Domain-Driven Design, Test-Driven Development.}
\vspace*{1em}

The purpose of this document is outline R\&D directions for advance the Trident Genesis platform.
There are 3 broad research categories, which are largely segregated from each other:

\begin{itemize}
    \item \textbf{Software Engineering}: modelling, automation, design and runtim tooling (verification, validation, visualisation), ;
    \item \textbf{User Interface and User Experience}: interaction with domain models, configurability, more advanced and smarter controls, notifications.
    \item \textbf{Integration}: GraphQL API.
\end{itemize}

\noindent And there are 3 broad categories that address cross-cutting concerns of the technology, which affect all of the above categories:

\begin{itemize}
    \item \textbf{Runtime Performance}: snappier UI, faster load times, optimised data queries and caching, more compile time optimisation.
    \item \textbf{Security}: improvements to application security, more refined read access control to entity properties, authenitcation and authorisation of 3rd party integrators.
    \item \textbf{Capabilities}: offline mode, scalability, async request process, task scheduling, auditing, even sourcing.
\end{itemize}


\section{Objectives}

