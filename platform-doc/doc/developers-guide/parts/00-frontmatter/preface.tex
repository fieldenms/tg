%%%%%%%%%%%%%%%%%%%%%%preface.tex%%%%%%%%%%%%%%%%%%%%%%%%%%%%%%%%%%%%%%%%%
%
% Use this file as a template for your own input.
%
%%%%%%%%%%%%%%%%%%%%%%%% Springer %%%%%%%%%%%%%%%%%%%%%%%%%%

\preface
  %\vspace{-100pt}
  Initially the Trident Genesis Platform (TG) was an attempt to automate a large number of repetitive and cumbersome programming tasks that we have personally  experienced over the course of several years developing EAM/ERP software using first Borland Delphi and later Java technologies in various combinations.
  The basic requirements for a platform was to provide a \emph{better} way for developing web-enabled software with a database back-end and rich GUI capabilities.
  Over the period of three years that took us to get TG to the level where it can be used for developing large scale applications, we have realised that what we're creating is not just a mere ``automation'', but a new technology for construction of the information systems in the domain-driven way.
  
  The development life cycle of TG-based applications revolves around the notion of the domain model, which serves as an information hub for other vital parts of the system such as the database, UI components and web communication.
  The domain model takes care of the validation, persistence, localisation, runtime state management and more.
  This approach simplifies not only the initial development of the system, but also significantly reduces the maintenance burden.

  The future platform releases will incorporate many new features and enhanced development concepts, which are currently in progress. 
  This includes things like special Eclipse plug-ins to further simplify development tasks and server-side load balancing to enhance the end-application scalability and performance.

  Hopefully, this book will not only guide the developer through the TG platform, but will also show a different and exciting way to create information systems.
  The reader should have a good understanding of the Java 5 programming language. Familiarisation with technologies such as JDBC, Servlet, Swing and Maven would be beneficial.
  
\vspace{\baselineskip}
\begin{flushright}\noindent
Lviv,\hfill ~\\
June 2011\hfill {\it Authors}\\
\end{flushright}


