\chapter{As Quickly as Possible or Maven Archetypes to the Rescue}\label{ch00:02}
%Divide and Conquer or Application Modules
\myepigraph{One must learn by doing the thing; for though you think you know it, you have no certainty, until you try.}{Sophocles}                                                   

  The Maven Archetype concept is a cool way of automating a lot of otherwise manual labour when creating new projects.
  \href{http://maven.apache.org/guides/introduction/introduction-to-archetypes.html}{Officially} archetype is a Maven project templating toolkit -- an original pattern or model from which all other things of the same kind are made.

  The Trident Genesis Platform provides an archetype that conveniently generates a project template, which then can be enhanced to meet customer requirements.
  The main goal of the provided archetype is to get developers up and running as quickly as possible with a simple, but fully operational TG-based application that would cover all main aspects of a \emph{complete} application.
  This includes brending, documentation, buid and deployment support.
  
  \begin{notebox}{Documentation.}{\label{mb:documentation-during-build}}
    Application documentation is built at the same time as the project, and it gets incorporated into the client application installation package.
    This way documentation becomes a first-class citizen of the development/deployment life cycle, and will never be forgotten during application releases.
    However, it does require \LaTeX to be installed on the system where application needs to be prepared for deployment.
  \end{notebox}

  The first section provides steps to use the Triden Genesis application archetype, and is followed by sections discussing the details of all the features of the generated project covering everything from importing the generated project into Eclipse down to project documentation and deployment.

\section{Create new project from an archetype}

  All commands for project generation are executed in a terminal\footnote{The majority of Maven related interaction is done via a terminal.
  Thus, it is useful be familiar with command line commands of your operating system of choice.
  All examples in this book are performed under Linux, but they are all fully applicable to Windows or Mac OS.
  Any Linux specific commands are highlighted in separate notes.}.
  Therefore, fire up the terminal and navigate to a working directory.
  For the sake of completeness, let's consider the that working directory is \texttt{workspace}, which is located in the user's home directory.
  
  If Maven is correctly configured as described in section \ref{ch00:01} then running command \texttt{mvn archetype:generate -Dfilter=fielden:} in a terminal should provide a result outlined in listing \ref{lst:find-and-run-archetype}.

  \lstset{  escapechar=\%,
	  morekeywords={mvn, archetype, generate, Dfilter, fielden, local},
	  numbers=left, numberstyle=\tiny, basicstyle=\scriptsize\color{black}, stepnumber=1, numbersep=5pt, keywordstyle=\bfseries\color{codefgcolor}, stringstyle=\color{stringcolor}}
  \begin{code}{Find and Run TG Application Archetype}{\label{lst:find-and-run-archetype}}{terminalbgcolor}
    \begin{lstlisting}
user@workstation:~/workspace$ mvn archetype:generate -Dfilter=fielden:
[INFO] Scanning for projects...
[INFO] Searching repository for plugin with prefix: 'archetype'.
[INFO] ------------------------------------------------------------------------
[INFO] Building Maven Default Project
[INFO]    task-segment: [archetype:generate] (aggregator-style)
[INFO] ------------------------------------------------------------------------
[INFO] Preparing archetype:generate
[INFO] No goals needed for project - skipping
[INFO] [archetype:generate {execution: default-cli}]
[INFO] Generating project in Interactive mode
[INFO] No archetype defined.
Choose archetype:
1: local -> fielden:tg-application-archetype (This is a project template for constructing 
	TG-based information systems. It provides templates for all application modules as 
	per Trident Genesis platform specification. The development teams should use this 
	template to significantly automates the initial creation of TG-based applications.)
Choose a number or apply filter (format: [groupId:]artifactId, case sensitive contains): : 
    \end{lstlisting}
  \end{code}

  This command requests Maven to generate a project using one of the archetypes with \texttt{groupId} equal to \emph{fielden}.
  The only available at this stage archetype associated with \emph{fielden} is \emph{fielden:tg-application-archetype}
  In order to select this archetype type 1 (or any other appropriate numeral, which points to archetype \emph{fielden:tg-application-archetype}) and hit \texttt{enter} to confirm the selection. 
  This action result in a number of archetype specific prompts. 
  Each of the prompts is illustrated and discussed below.
  It is best if reader follow this section step by step trying to reproduce each of them on their system.

  \begin{enumerate}
    \item Define value for property 'groupId': : -- a Maven specific variable, which identifies the group ID of the project being created; this could be a short name (without spaces or other funny characters) of your company such as \emph{fielden}; hit \texttt{enter} to confirm.
      
    \begin{code}{Application groupId}{\label{lst:archetype-groupId}}{terminalbgcolor}
      \begin{lstlisting}
	Define value for property 'groupId': : fielden		
      \end{lstlisting}
    \end{code}

    \item Define value for property 'artifactId': : -- a Maven specific variable, which identifies the artifact ID of the project being created; this could be a short name (without spaces or other funny characters) of your project such as \emph{coolapp}; note that this value is used as part the project's top level directory naming structure; hit \texttt{enter} to confirm.
    
    \begin{code}{Application artifactId}{\label{lst::archetype-archetypeId}}{terminalbgcolor}
      \begin{lstlisting}
	Define value for property 'groupId': : fielden		
	Define value for property 'artifactId': : coolapp
      \end{lstlisting}
    \end{code}

    \item Define value for property 'version': 1.0-SNAPSHOT: -- a Maven specific variable, which identifies version of the artifact (i.e. your project); the default is \emph{1.0-SNAPSHOT}, which is only reasonable given a new project is being created; hit \texttt{enter} to confirm.
    
    \begin{code}{Application version}{\label{lst::archetype-version}}{terminalbgcolor}
      \begin{lstlisting}
	Define value for property 'groupId': : fielden		
	Define value for property 'artifactId': : coolapp
	Define value for property 'version': 1.0-SNAPSHOT:
      \end{lstlisting}
    \end{code}

    \item Define value for property 'package': fielden: -- a Maven specific variable, which identifies Java package name, which is used as the root for all generated Java files; the default value matches the groupId, which is reasonable due to a Java custom of naming packages based on the company name (actually Internet domain name, which in most cases should correspond to the company name); hit \texttt{enter} to confirm.
    
    \begin{code}{Application default package}{\label{lst::archetype-package}}{terminalbgcolor}
      \begin{lstlisting}
	Define value for property 'groupId': : fielden		
	Define value for property 'artifactId': : coolapp
	Define value for property 'version': 1.0-SNAPSHOT:
	Define value for property 'package': fielden:
      \end{lstlisting}
    \end{code}

    \item Define value for property 'companyName': : -- a TG archetype specific variable, which should be provided with a full name of the company, which will be delivering the project being generated; the example below uses value \emph{Fielden Management Services}; hit \texttt{enter} to confirm.
    
    \begin{code}{Application default package}{\label{lst::archetype-package}}{terminalbgcolor}
      \begin{lstlisting}
	Define value for property 'groupId': : fielden		
	Define value for property 'artifactId': : coolapp
	Define value for property 'version': 1.0-SNAPSHOT:
	Define value for property 'package': fielden:
	Define value for property 'companyName': : Fielden Management Services
      \end{lstlisting}
    \end{code}

    \item Define value for property 'prjectName': : -- a TG archetype specific variable, which should be provided with a full name of the project; the example below uses value \emph{Coo App}; hit \texttt{enter} to confirm.
    
    \begin{code}{Application default package}{\label{lst::archetype-package}}{terminalbgcolor}
      \begin{lstlisting}
	Define value for property 'groupId': : fielden		
	Define value for property 'artifactId': : coolapp
	Define value for property 'version': 1.0-SNAPSHOT:
	Define value for property 'package': fielden:
	Define value for property 'companyName': : Fielden Management Services
	Define value for property 'prjectName': : Cool App
      \end{lstlisting}
    \end{code}

    \item Define value for property 'prjectWebSite': : -- a TG archetype specific variable, which should be provided with an URI to the project web site; the value is required, so if there is no project web site then a fictitious one should be specified; please note that it is required to specify access protocol as part of the URI such as \texttt{http://}; hit \texttt{enter} to confirm.
    
    \begin{code}{Application default package}{\label{lst::archetype-package}}{terminalbgcolor}
      \begin{lstlisting}
	Define value for property 'groupId': : fielden		
	Define value for property 'artifactId': : coolapp
	Define value for property 'version': 1.0-SNAPSHOT:
	Define value for property 'package': fielden:	
	Define value for property 'companyName': : Fielden Management Services
	Define value for property 'prjectName': : Cool App
	Define value for property 'prjectWebSite': : http://www.fielden.com.au/coolapp
      \end{lstlisting}
    \end{code}

    \item Define value for property 'supportEmail': : -- a TG archetype specific variable, which should be provided with an intended for this project support email address; hit \texttt{enter} to confirm.
    
    \begin{code}{Application default package}{\label{lst::archetype-package}}{terminalbgcolor}
      \begin{lstlisting}
	Define value for property 'groupId': : fielden		
	Define value for property 'artifactId': : coolapp
	Define value for property 'version': 1.0-SNAPSHOT:
	Define value for property 'package': fielden:	
	Define value for property 'companyName': : Fielden Management Services
	Define value for property 'prjectName': : Cool App
	Define value for property 'prjectWebSite': : http://www.fielden.com.au/coolapp
	Define value for property 'supportEmail': : coolapp@support.fielden.com.au
      \end{lstlisting}
    \end{code}

  \end{enumerate}

  Once the above variables have been entered Maven lists all of their values and prompts the developer to confirm their values before generating the project. 
  The default confirmation value is \texttt{Y} -- pressing \texttt{enter} leads to confirmation of the values' correctness. 
  Typing \texttt{N} and pressing \texttt{enter} leads to new prompts for all the variables.

  
  \begin{code}{Properties configuration confirmation}{\label{lst:properties_confirmation}}{terminalbgcolor}
      \begin{lstlisting}
	Confirm properties configuration:
	groupId: fielden
	artifactId: coolapp
	version: 1.0-SNAPSHOT
	package: fielden
	companyName: Fielden Management Services
	projectName: Cool App
	projectWebSite: http://www.fielden.com.au/coolapp
	supportEmail: coolapp@support.fielden.com.au
	Y:
      \end{lstlisting}
  \end{code}

  Confirming properties should result in a successful generation of the TG-based application.
  Alternatively, it could be more convenient to avoid the interactive prompts and run TG archetype in a single command as shown in listing~\ref{lst:archetype_single_command}.

  \begin{code}{Single command for project generation}{\label{lst:archetype_single_command}}{terminalbgcolor}
      \begin{lstlisting}
  mvn archetype:generate \
	-DarchetypeGroupId=fielden \
	-DarchetypeArtifactId=tg-application-archetype \
	-DarchetypeVersion=1.0 \
	-DgroupId=fielden \
	-DartifactId=coolapp \
	-Dversion=1.0-SNAPSHOT \
	-Dpackage=fielden \
	-DcompanyName="Fielden Management Servces" \
	-DprojectName="Cool App" \
	-DprojectWebSite=http://www.fielden.com.au/coolapp \
	-DsupportEmail=coolapp@support.fielden.com.au
      \end{lstlisting}
  \end{code}

  Note \ref{mb:project-dir-structure} lists a directory structure of the generated application skeleton.
  The generated project is a complete TG-based application, which is ready to be built and deployed.
  All of this is achieved by running just one command, significantly stream lining the process of project initiation.
 
  \begin{notebox}{Inspect project directory structure.}{\label{mb:project-dir-structure}}
      Under Linux a directory structure of the generated project can be inspected by using the \texttt{tree} command as follows.
	
      \begin{verbatim}
	user@workstation:~/workspace$ tree -L 2 -d
	.
	|-- coolapp
	|   |-- coolapp-dao
	|   |-- coolapp-pojo-bl
	|   |-- coolapp-rao
	|   |-- coolapp-ui
	|   |-- coolapp-web-client
	|   `-- coolapp-web-server
      \end{verbatim}
  \end{notebox}

  Before delving into the details of enhancing the generated project with additional functionality that would address the business requirements, the next section outlines the build and deployment process.
  This process is identical for every TG-based application, and once learned it will allways be easily reused.

\section{Build, Deploy, Run}
  Maven is used to fully manage the project, which includes its build and deployment process.
  In order to follow the process please navigate to the generated project directory in the terminal as illustrated in listing~\ref{lst:cd_to_project}.
  
  \begin{code}{Navigate to the project main directory}{\label{lst:cd_to_project}}{terminalbgcolor}
      \begin{lstlisting}
	user@workstation:~/workspace$ cd coolapp
        user@workstation:~/workspace/coolapp$
      \end{lstlisting}
  \end{code}

  Then run the Maven \texttt{clean} and \texttt{install} goals as illustrated in listing~\ref{lst:mvn_clean_install}, which generates all application deployment artifacts.
  The \texttt{clean} goal is not essential for the first run, but in order not to think when it is needed and when not, simply always use it.

  \begin{code}{Project installation}{\label{lst:mvn_clean_install}}{terminalbgcolor}
     \begin{lstlisting}
        user@workstation:~/workspace/coolapp$mvn clean install
     \end{lstlisting}
  \end{code}

  Successful execution of this command should complete with an output similar to the listed in listing~\ref{lst:mvn_clean_install_completed}.
  The build process produces two artifacts -- an application web server (unpacked and ready to be used) and a client application installation package.
  
  \begin{code}{Successful project installation output}{\label{lst:mvn_clean_install_completed}}{terminalbgcolor}
      \begin{lstlisting}
[INFO] ------------------------------------------------------------------------
[INFO] Reactor Summary:
[INFO] ------------------------------------------------------------------------
[INFO] Cool App Parent Project ............................... SUCCESS [4.212s]
[INFO] Cool App POJOs and Business Logic Module .............. SUCCESS [6.380s]
[INFO] Cool App DAO Module ................................... SUCCESS [12.244s]
[INFO] Cool App UI Module .................................... SUCCESS [6.673s]
[INFO] Cool App RAO Module ................................... SUCCESS [2.736s]
[INFO] Cool App Web Client Module ............................ SUCCESS [18.680s]
[INFO] Cool App Web Server Module ............................ SUCCESS [3.523s]
[INFO] ------------------------------------------------------------------------
[INFO] ------------------------------------------------------------------------
[INFO] BUILD SUCCESSFUL
[INFO] ------------------------------------------------------------------------
[INFO] Total time: 55 seconds
[INFO] Finished at: Mon Mar 12 13:05:07 EET 2012
[INFO] Final Memory: 81M/416M
[INFO] ------------------------------------------------------------------------
user@workstation:~/workspace/coolapp$ 
      \end{lstlisting}
  \end{code}

\subsection{Server}

  The application server is represented by the \emph{Cool App Web Server Module}, which is located in directory \texttt{coolapp-web-server}.
  The build process compiles project modules and automatically collects all dependencies, executions scripts, a sample database, configuration into the deployment directory \texttt{coolapp-web-server/target/deployment}.
  It also packages this directory as a zip arhive file \texttt{coolapp-web-server-1.0-SNAPSHOT-distribution.zip} located under \texttt{coolapp-web-server/target}, which is ready for distibution.
  Listing \ref{lst:app_server} outlines the structure of this directory.

  \begin{code}{Application server directories}{\label{lst:app_server}}{terminalbgcolor}
     \begin{lstlisting}
user@workstation:~/workspace/coolapp$ tree -L 1 coolapp-web-server/target/deployment/
coolapp-web-server/target/deployment/
|-- application.properties
|-- attachments
|-- client
|-- db
|-- lib
|-- log4j.xml
|-- start-server.sh
|-- start-server.bat
`-- webapp
     \end{lstlisting}
  \end{code}

  Navigate to the deployment directory\footnote{This is important in order for the start-up script to correctly set the application directory.} and simply run the \texttt{start-server.bat} under Windows or \texttt{start-server.sh} under Linux in order to start an application server.
  Once started it listens on port \texttt{8091} for incoming connections.
  Now it's time to install the client application.

\subsection{Client}
  \sloppypar
  If an application server is up and running then a new terminal window (or a tab) should be open to start client application installation.
  The client application is represented by module \emph{Cool App Web Client Module}, which is located in directory \texttt{coolapp-web-client}.
  As part of the build process, an installation package is assembled and placed into directory \texttt{coolapp-web-client/target} as an executable jar file \texttt{coolapp-web-client-1.0-SNAPSHOT-standard.jar}\footnote{The \texttt{1.0-SNAPSHOT} indicates the current project version, which is used by default during generation.}.   
  This jar file can be executed from a file manager by double clicking on it\footnote{If JRE is properly installed then Java should be associated} or from a terminal by running the \texttt{java} process as outlined in listing~\ref{lst:install_client}

  \begin{code}{Application server directories}{\label{lst:install_client}}{terminalbgcolor}
     \begin{lstlisting}
user@workstation:~/workspace/coolapp$ java -jar \
> coolapp-web-client/target/coolapp-web-client-1.0-SNAPSHOT-standard.jar
     \end{lstlisting}
  \end{code}

  The first screen of the installation process is depicted on Fig.~\ref{img:ch00:02:client_installation_first_screen}.
  Please note the information on this screen, which reuses the project properties that were provided as part of project generation.
  For example, it states that the software is developed by \emph{Fielden Management Services} and also lists application support email address and a web page.
  
  Simply follow the steps of the installation process, which should result in successful deployment of the client application.
  One of the final steps lists components that are provided as part of the installation.
  Specifically, \emph{Main Application} and \emph{Documentation} as depicted in Fig.~\ref{img:ch00:02:client_installation_doc}.
  An interesting thing here is the documentation, which is an optional component (can be unchecked).
  Application documentation gets generated during the build process.
  Its creation and maintenance is discussed later in this chapter.

  \begin{image}{First screen of the client installation process}{\label{img:ch00:02:client_installation_first_screen}}    
    \includegraphics[width=0.6\textwidth]{parts/00-part/chapters/01-application-modules/images/01-client-installation.png}
  \end{image}

  \begin{image}{Client documentation provided as part of the installation}{\label{img:ch00:02:client_installation_doc}}    
    \includegraphics[width=0.6\textwidth]{parts/00-part/chapters/01-application-modules/images/02-client-installation-doc.png}
  \end{image}

  Another important screen is where user needs to specify an application server URI and port.
  The default value is \url{www.fielden.com.au}, but for our purpose due to locally deployed application server, it should be changed to \texttt{localhost} as depicted in Fig.~\ref{img:ch00:02:client_installation_uri}.

  \begin{image}{Application server URI}{\label{img:ch00:02:client_installation_uri}}    
    \includegraphics[width=0.6\textwidth]{parts/00-part/chapters/01-application-modules/images/03-client-installation-uri.png}
  \end{image}
  
  Once installation is completed, run the client application by either navigating to the specified installation directory and running \texttt{run.bat} (\texttt{run.sh} for Linux) or by choosing an appropriate program from the \emph{Start} menu under Windows.
  The login screen of the generated application is depicted in Fig.~\ref{img:ch00:02:client_login}.
  The default user name and password is \emph{SU} (both capital).

  \begin{image}{Application login}{\label{img:ch00:02:client_login}}    
    \includegraphics[width=0.6\textwidth]{parts/00-part/chapters/01-application-modules/images/04-client-login.png}
  \end{image}

  The main window of the launched client application is depicted on Fig.~\ref{img:ch00:02:client_main_window}.
  As can be observed, the generated application provides all the necessary functionality for managing users, roles, security tokens as well as the default implementation for \emph{Personnel} and \emph{Attachments}.

  \begin{image}{Application main window}{\label{img:ch00:02:client_main_window}}    
    \includegraphics[width=0.7\textwidth]{parts/00-part/chapters/01-application-modules/images/05-client-main-window.png}
  \end{image}

  The above demonstrated how easy it is to start building a TG-based application.
  The provided Trident Genesis Archetype generates a fully operational software with ready to be distributed application server deployment files and a client installation package.
  The use of command line tools for building the project is important as it can be used on headless build servers, which a common practice in the software industry.

  It is important to note that Trident Genesis Platform offers a complete development methodology, which includes the unified programming model, software distribution model and application documentation creation.
  
  The following sections provide more detailed explanation about the generated application modules, how to set up a development environment and make simple project changes.