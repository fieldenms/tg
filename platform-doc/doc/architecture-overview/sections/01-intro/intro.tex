% need to include an image of the business rules to mind to code transition
\section{Motivation}\label{sec:01}
  Software information systems\footnote{In this article the term \emph{information system} is used interchangably with term \emph{software application}.} (IS) are ubiquitous in the modern society, which makes the ability to create and maintain reliable software critical from both consumer and manufacturer perspectives.
  The ideas and facilities built into a software application fully determine its purpose and value to the consumper (end user).
  That's why the subject of this article are the technological innovations, which combined define a unique technology for the development of reliable business oriented information systems -- the Trident Genesis Platform (TG).

  First of all, it is important to understand what motivated the creation of TG and thus determined its main purpose.
  The evolution of computers lead to increase of computational power, which facilitated a significant incease of the amount of data that can be processed, and at the same time led to a need for higher level of abstractions when creating software applications.
  Programming languages are at the core of software development, which also evolved -- from machine and assembly languages to modern mainstream programming languages such as Java, Scala, C\# etc.
  However, due to their nature all programming langages are convenient for instructing the computing machines how to perform specific computations, but not convetient for decribing (modeling) the actual problem domain of the business being automated\footnote{Business for which an information system is being created.}.
  
  This leads to a huge semantic gap between the business requriements (what needs to be solved) and the actual solution (information system), which constitutes itself as a set of instruction to the computer.
  The two sides of the gap are briged only by the software developer(s) who has built the solution and holds the trancition model in their mind.
  This and due to the fact that the same business requirement can be implemented in multiple ways using a general purpose programming language, there are lot of problems maintaining a solution especially when the original developer(s) are not around.

  In his ``Language Oriented Programming'' M.~P.~Ward points to several important research results:
  \begin{itemize}
    \item There is a thin spread of domain knowledge among software developers in most projects;    
    \item Most development is maintenance. 
	  System evolution is so common, that a development from scratch is the exception rather than the rule;
    \item Most specification is incremental. 
	  The customer is rarely able to provide a complete specification at any stage of the project;
    \item Domain knowledge is important;
    \item There is a gulf between developer and user. 
	  Few developers had adequate knowledge about the user’s work. 
	  This led to major misunderstandings about the system’s purpose;
  \end{itemize}
  
  To date most of the development of business applications involves low technical details, which deprive developers from the time to think and work on the actual business requirements.
  
  The above problems and our own experience of having to deal with them on a daily basis led to the need of increasing the level of abstration that would hide low technical details and provide a uniform programming model for implementing software solutions as close as possible to the business terminology.
