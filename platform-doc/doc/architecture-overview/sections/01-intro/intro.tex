% need to include an image of the business rules to mind to code transition
\section{Motivation -- the ``why'' behind Triden Genesis}\label{sec:01}
  Software information systems\footnote{In this article the term \emph{information system} is used interchangeably with term \emph{software application}.} (IS) are ubiquitous in the modern society, which makes the ability to create and maintain reliable software critical from both consumer and manufacturer perspectives.
  The ideas and facilities built into a software application fully determine its purpose and value to the consumer (end user).
  That's why the subject of this article are the technological innovations, which combined define a unique technology for the development of reliable business oriented information systems -- the Trident Genesis Platform (TG).

  \begin{wrapfigure}{r}{80mm}
    \centering    
    \begin{tikzpicture}[mindmap, opacity=0.8, scale=0.5, outer sep=0pt, transform shape]

      \begin{scope}[mindmap, concept color=green!50!white, text=green!50!black, level 1 concept/.append style={level distance=130, sibling angle=35}]
	\node [concept] at (-1.0, 2.0) (is) {Information Systems}[clockwise from=200] 
	  child {node [concept] (log) {Domain Models}}
	  child {node [concept] (alg) {Data Management}}
	  child {node [concept] (log) {User Interface}}
	  child {node [concept] (img) {Web Communication}}
	  child {node [concept] (opt) {Analysis and Reporting}}
	  child {node [concept] (res) {Security}};
      \end{scope}

      %\node (dev) at (2.1,-4) [circle, minimum size=4cm,fill,draw,thick,color=red!50, text=red!50!black] {Developers};
      \begin{scope}[mindmap, concept color=red!50, text=red!50!black, level 1 concept/.append style={level distance=150, sibling angle=30}]
	\node [concept] at (3.1,-4) (dev) {Developers}[clockwise from=70] 
	  child {node [concept] (ide) {IDE}}
	  child {node [concept] (req) {Requirements Management}}
	  child {node [concept] (dom) {Domain Modelling}}      
	  child {node [concept] (src) {Source Version Control}};
      \end{scope}

      \begin{scope}[mindmap, concept color=blue!50, text=blue!50!black, level 1 concept/.append style={level distance=130, sibling angle=35}]
	\node [concept] at (0.2,-10.0) (bus) {Business Requirements}[clockwise from=-5] 
	  child {node [concept] (accbud) {Accounting and Budgeting}}
	  child {node [concept] (ship) {Shipments}}
	  child {node [concept] (ord) {Ordering Products}}
	  child {node [concept] (work) {Work Orders}}      
	  child {node [concept] (hum) {Human Resources}};
      \end{scope}
    % 
    % Connections 
      \path (bus) to[circle connection bar switch color=from (blue!50) to (red!50)] (dev);
      \path (dev) to[circle connection bar switch color=from (red!50) to (green!50)] (is);
    \end{tikzpicture}
  \end{wrapfigure}

  First of all, it is important to understand what motivated the creation of TG and thus determined its main purpose.
  The evolution of computers lead to increase of computational power, which facilitated a significant increase of the amount of data that can be processed, and at the same time led to a need for higher level of abstractions when creating software applications.
  Programming languages are at the core of software development, which also evolved -- from machine and assembly languages to modern mainstream programming languages such as Java, Scala, C\# etc.
  However, due to their nature all programming languages are convenient for instructing the computing machines how to perform specific computations, but not convenient for describing (modelling) the actual problem domain of the business being automated\footnote{Business for which an information system is being created.}.
  
  This leads to a huge semantic gap between the business requirements (what needs to be solved) and the actual solution (information system), which constitutes itself as a set of instruction to the computer.
  The two sides of the gap are bridged only by the software developer(s) who has built the solution and holds the transition model in their mind.
  So, the software developers serve as translators between the language of the business domain and the language of the software information system.
  Due to this and the fact that the same business requirements can be expressed in multiple ways using a general purpose programming language, there is a lot of problems maintaining the solution especially when the original developer(s) are not around.

  In his ``Language Oriented Programming'' M.~P.~Ward points to several important research results:
  \begin{itemize}
    \item There is a thin spread of domain knowledge among software developers in most projects;    
    \item Most development is maintenance. 
	  System evolution is so common, that a development from scratch is the exception rather than the rule;
    \item Most specifications are incremental. 
	  The customer is rarely able to provide a complete specification at any stage of the project;
    \item Domain knowledge is important;
    \item There is a gulf between developer and user. 
	  Few developers have adequate knowledge about the user's work. 
	  This leads to major misunderstandings about the system's purpose.
  \end{itemize}
  
  To date most of the development of business applications involves low technical details, which deprive developers from the time to think and work on the actual business requirements.
  
  The above problems and our own experience of having to deal with them on a daily basis led to the need of raising the level of abstraction that would hide low technical details and provide a uniform programming model for implementing software solutions as close as possible to the business terminology.
  By raising the level of abstraction the TG platform brings closer together software developers and business domain specialist, which is one of the platform's less obvious but defining features.
