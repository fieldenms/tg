\section{Data Management \& Web Resources}\label{sec:04}

  One of the central differentiators of business applications from other information systems is their paradigm for data processing.
  Data related activities are the most important and at the same time the most vulnerable part of business applications.
  Vulnerability can be explained by the lack of~~``close to perfection'' solutions, unlike for some other types of software systems such as word processing or digital painting software.
  Business applications are constantly under pressure from contradicting requirements.
  On the one hand, there is a need for processing large datasets, while on the other hand business applications should demonstrate good performance and rich functionality.
  The amount of data to be processed always increases, requirements for more diverse functionality occur constantly and the need for high scalability of business applications is ever increasing.
  
  There are several well accepted data processing paradigms currently in use for developing business applications.
  However, there is no single best solution that would satisfy all business requirements.
  It can be observed that most if not all available solutions implement different combinations of data management paradigms, which represent compromises to improve certain specifically targeted characteristics.

  With the increase in the role of the Internet as a communication mechanism for enterprises as well as the maturity of many Internet technologies, business applications are required to process data not only by directly communicating with databases\footnote{Usually RDBMS.}, but also from distributed across the web resources.
  This introduces an additional layer of complexity, where in most cases developers need to switch between paradigms of working with relational databases and with web resources.

  The Trident Genesis platform introduces its own data processing solution by reusing some of the existing paradigms in its own unique way to provide a uniform programming model for data processing against both databases and web resources\footnote{Web resources developed using the Trident Genesis platform.}.
  All data CRUD (create, request\footnote{Commonly, ``R'' stands for \emph{read} or \emph{retrieve}, but here we'd like to emphasise a more rich functionality behind the ``R'' in TG as described later in the text.}, update, delete) operations utilise a pure object-oriented approach combined with a carefully designed internal DSL called ``Entity Query Language''.
  This means that developers do not operate on database records or web resources directly.
  Instead they are confined to the much higher conceptual level of business domain models and the query language that operates directly on those models.
  
  For example, in order to modify persisted data there is no need to write complex low level queries and translate their result into the business domain model.
  Persisted business entities are requested declaratively, modified by changing their properties at the object level and then saved declaratively.
  The platform fully takes care of handling low-level technical details for translating declaratively expressed actions into corresponding web or database API calls.
  This development model provides a convenient way to implement business rules related to changes of domain entities such as validation, and before and after save logic, which can lead to modification of additional entities if required.
  It also fully supports both \emph{remote} and \emph{local}\footnote{The \emph{remote} here means the server-side application, and \emph{local} -- the client-side application running on users' local machines.} data processing, which provides developers with a flexible way to leverage computational resources by controlling how ``thin'' their business client and server applications should be.

  \subsection{Client- \& Server-Side Applications}
  With the advent of multi-tier software architecture, the three-tier architecture became a de-facto standard for developing web-enabled applications.
  The current state and trend in Web development technologies is twofold: on one side there is an ``in-browser HTML'' model, where client applications are served as web pages, on another -- the rich client application model, where client applications are fully capable desktop applications.
  Both models have their strengths and weaknesses.

  As mentioned above, the programming model enforced by the platform strives for uniformity. 
  The platform follows the path of reducing the number of concepts and technologies needed for the development of business applications.
  One of the key aspects of this, is the use of Java as the only programming language required to develop business applications with TG.
  This is achieved by utilising the Rich Internet Application (RIA) paradigm for developing both client and server applications using only the Java programming language.
  
  From a practical perspective of software development this has a significant advantage over the ``in-browser HTML'' model.  
  For example, such important activities as debugging and profiling are performed in the familiar environment of the preferred IDE, where developers think in terms of their primary programming language instead of handling the complexity of context switching between disparate languages\footnote{For example, Java for the server application and JavaScript for the client application.}.
  At the same time, the end-users of business applications benefit from the power of the fully fledged JVM utilised by the client-side application to provide high performance and an excellent usability experience.
  
  The server-side represents a set of loosely coupled web resources that adhere to principles of the Resource Oriented Architecture (ROA).
  This architectural style lends itself very well to developing scalable business applications, which can leverage different deployment infrastructures -- from a single server machine to a cluster of multiple machines and the cloud.
  The provided development model hides the technical complexity of developing web resources\footnote{The platform supports the development of custom web resources if required, but such a need is unlikely due to the rich semantics of the provided domain level abstractions.} by treating all domain entities as such.
  The platform understands the application execution context, automatically choosing the appropriate handling mechanism of domain entities either as ordinary Java objects in the local memory or as web resources residing on the server side.
  
  For example, if the business application attempts to save a domain entity\footnote{It could either be a persistent or synthetic entity, which is possible due to the platform's uniform development model.}, the underlying platform mechanism would determine the origin of the request, resulting in either a call to the database for the server-side application or a corresponding Web resource for the client-side application.
  All of this is accompanied by automatic transaction demarcation ensuring referential integrity of the data.

   \begin{figure}[!h]
    \centering    
    \begin{tikzpicture}[>=latex']
      \tikzset{
	  outercore/.style={circle, fill=blue!50!white, inner sep=0em, minimum size=0.6cm},
	  core/.style={circle, shade, ball color=green!50!white, inner sep=0em, minimum size=0.3cm},
	  score/.style={circle, fill=green!50!black, inner sep=0em, minimum size=0.3cm},
	  outer/.style={circle, fill=blue!50!white, inner sep=0em, minimum size=2.3cm},
	  inner/.style={circle, fill=green!50!white, inner sep=0em, minimum size=1.5cm},
	  trans/.style={circle, fill=yellow!50!orange, inner sep=0em, minimum size=2.7cm}
      }

      %-----#1 height, #2 width, #3 aspect, #4 name of the node, #5
      %-----coordinate, #6 label
      \def\aboxl[#1,#2,#3,#4,#5]#6{%
	\node[draw, cylinder, alias=cyl, shape border rotate=90, aspect=#3, %
	minimum height=#1, minimum width=#2, outer sep=-0.5\pgflinewidth, %
	color=orange!60!black, left color=yellow!80, right color=yellow!80!orange, middle
	color=white] (#4) at #5 {};%
	\node at #5 [orange!80!black] {#6};%
	\fill [yellow!30] let \p1 = ($(cyl.before top)!0.5!(cyl.after top)$), \p2 =
	(cyl.top), \p3 = (cyl.before top), \n1={veclen(\x3-\x1,\y3-\y1)},
	\n2={veclen(\x2-\x1,\y2-\y1)} in (\p1) ellipse (\n1 and \n2); }
      
      \begin{scope}[pattern=dots]
	\node (o) at (0, -0.25) [outer, opacity=0.8][postaction={pattern=north west lines,pattern color=blue!20}] {};      	
	\node (i) at (0, -0.25) [inner, opacity=0.8,][postaction={pattern=north west lines,pattern color=green!20}] {} node [below,text=blue!50!black,yshift=-1.7cm] {Client-Side};

	\begin{scope}[scale=0.3]
	  \node (t) at (0,0) [outercore, opacity=0.8][postaction={pattern=north west lines,pattern color=green!20}] {};	
	  \node at (0,0) [core, opacity=0.5][postaction={pattern=north west lines,pattern color=green!20}] {};

	  \node (r) at (1,-1.2) [outercore, opacity=0.8][postaction={pattern=north west lines,pattern color=green!20}] {};	
	  \node at (1,-1.2) [core, opacity=0.5][postaction={pattern=north west lines,pattern color=green!20}] {};

	  \node (l) at (-1,-1.2) [outercore, opacity=0.8][postaction={pattern=north west lines,pattern color=green!20}] {};	
	  \node at (-1,-1.2) [score][postaction={pattern=north west lines,pattern color=green!20}] {};
	\end{scope}
      \end{scope}

      \begin{scope}[xshift=8cm]       
	\node[trans] at (0, -0.25) {} node [above,text=orange!80!black, scale=0.7,yshift=1.7cm] {Transaction Demarcation};;
	\node (o2) at (0, -0.25) [outer, opacity=0.8] {};
	\node (i2) at (0, -0.25) [inner, opacity=0.8] {} node [below,text=blue!50!black,yshift=-1.7cm] {Server-Side};

	\begin{scope}[scale=0.3]
	  \node (t2) at (0,0) [outercore, opacity=0.8] {};	
	  \node (t2c) at (0,0) [core, opacity=0.5] {};

	  \node (r2) at (1,-1.2) [outercore, opacity=0.8] {};	
	  \node (r2c) at (1,-1.2) [core, opacity=0.5] {};

	  \node (l2) at (-1,-1.2) [outercore, opacity=0.8] {};	
	  \node at (-1,-1.2) [score] {};
	\end{scope}	
      \end{scope}     
      \begin{scope}[xshift=3.93cm,yshift=-0.3cm]
	\node[shape=double arrow,draw,minimum height=9cm,scale=0.6] {CRUD Operations};
      \end{scope}
      \begin{scope}[xshift=12cm,yshift=-0.5cm]
	\aboxl[2.0cm,1.5cm,1.6,a1,(0,0)] {DB};
	\node [xshift=-0.6cm, yshift=0.5cm] (mark1) {};
	\node [xshift=-0.6cm, yshift=-0.3cm] (mark2) {};
      \end{scope}


      \fill [very thick, green!50!black,->,out=30,in=160] (t2c.east) edge (mark1.west);      
      \fill [very thick, green!50!black,->,out=-30,in=-160] (r2c.east) edge (mark2.west);
    \end{tikzpicture}   
  \end{figure}


  \subsection{Data Communication and Processing}
  
  Despite a recent surge of NoSQL databases, the vast majority of Enterprise Applications depend on Relational Database Systems that provide the most adequate solution for managing data.
  In fact, many researches show that NoSQL capabilities are mainly applicable in the domains where there is a lot of poorly structured data such as documents, while Enterprise domains are best represented with a proper model that captures structural aspects and relational dependencies.
  Object-oriented programming languages serve as the most popular choice for developing Enterprise Applications that require efficient ways to communicate with relational databases.  
  Current ORM solutions, which are used to resolve object-relational impedance mismatch, provide significantly less powerful query mechanisms than existing RDBMS systems.
  The result of our research and development provides a way to overcome many limitations of data querying facilities offered by modern ORM frameworks.
  
  \subsubsection{ORM Approach}
  The main emphasis in modern ORM frameworks is heavily shifted towards approaches for the best mapping of object-oriented concepts such as classes, inheritance, encapsulation to the relational concept (sets and relations).
  This area is well developed and there are industry wide initiatives for producing an API standard similar to SQL.
  One of such standards is Java Persistence API (JPA), which has a vibrant community and dynamic adaptation by many vendors.
  
  However, there is still a lot of debate about advantages and disadvantages of ORM solutions.
  One of the weakest aspect of ORM remains its limitation of data querying capabilities that are significantly behind of its relational counter part -- Structured Query Language.
  We believe that the reason for this lies in the way Enterprise Systems are structured:
  \begin{itemize}
   \item Front-end applications for the majority of users, which are frequently build using OOP languages and thus requiring an ORM framework. 
   Usually, such applications have rather rudimentary support for analytical features.
   \item External business intelligence tools that provide rich analytical capabilities, but completely ignore the application layer (i.e. the domain model expressed in object-oriented terms), and instead work directly with the underlying database.
  \end{itemize}
  Existing technologies such as Hibernate, EclipseLink, Microsoft Entity Framework have a limited data querying support.
  And while ORM frameworks try to hide relational nature of databases, their querying mechanisms still very closely resemble relational operators from SQL.  
  
  The holistic approach to information system modelling is essential for making sustainable applications that could fulfil the needs of different groups of users.
  One of the evident proofs of this is the emerging concept of \emph{Embedded Business Intelligence} that brings analytical capabilities back to the core system.
  This concept offers instant decision making and immersive user experience, and its implementation requires a holistic information system design with proper domain modelling and information retrieval function.
  A fully capable data querying mechanism is an inherent building block of such systems.
  In our view this should be achieved by close integration between the domain model and the querying mechanism that would allow tapping into the knowledge it captures.
  
   \subsubsection{Entity Query Language}  
  The platform ensures effective technological support of the object-oriented programming approach, which underpins the provided development model.
  One of the principal solutions incorporated into the platform is the \emph{Entity Query Language} -- a declarative query language based on SQL that provides a number of significant enhancements. 
  The primary objective of these enhancements is the support for working with business entities in a natural (for object-oriented programming) way, which ensures effective and ergonomic means for implementing business solutions.
  
  One of the key benefits of the object-relational paradigm for data management is its clarity and simplicity for developing applications.
  Object-orientation provides excellent readability of the algorithms implementing the business logic, significantly reduces the number of programming errors and ensures data integrity at a high level.
  
  Entity Query Language (EQL) represents an internal Java DSL, which embraces the fluent interface concept and object composition to provide a type safe query language that matches the data processing power of SQL, but exceeds it in clarity and expressiveness.
  This provides a natural way to express data processing instructions in terms of the business domain instead of the low level data structures.
  The implementation of EQL as an internal DSL automatically takes advantage of modern Java IDEs, which provide code highlighting, code insight etc.
  This favourably differentiates EQL from alternatives such as Hibernate Query Language where queries are expressed as strings without any semantics for an IDE or compiler to validate them at design time.
  Thus, every EQL query is validated at design/compile time naturally fitting into Java's statically typed nature.

  \paragraph{Object graph traversal.}
  Object composition serves as the way for modelling relationships between domain entities in OOP.
  Thus, if it is necessary to access an object property in the context of some other object the notion of \emph{dot-notated property path} is used, where the dot symbol (aka \emph{navigation operator}) separates properties from each other.
  Dot-notated paths represent paths in object graphs stored in computer memory and provide an essential mechanism for accessing data represented as objects.
  Therefore, traversal of dot-notated paths from the relational perspective is one of the key aspect for data querying.
  
  Current ORM frameworks translate dot-notated paths into a series of INNER JOIN operators, which are applicable only to non-collectional composition\footnote{Composition representing only many-to-one and one-to-one associations, many-to-zero and one-to-zero are also not supported.}.
  Our model for relation traversal supports automatic selection of the most appropriate strategy -- INNER JOIN, OUTER JOIN, EXISTS and their combination.
  This is achieved by close integration with the underlying domain model and its metadata in conjunction with high level query language (Entity Query Language).
  
  Another important advancement supported by the developed query mechanism is the transparent use of collectional properties (i.e. one-to-many relationship).
  This provides a proper object-oriented approach to data query unlike existing ORM solutions that enforce the explicit use of set operators to achieve the same result.
  The following example illustrates some of the described features.  
  This query retrieves all Mercedes vehicles that consumed more UNLEADED fuel than PREMIUM over their lifetime.

  \lstset{morekeywords={val,select,where,prop,eq,and,begin,end,starts_with,any_of_values,values,or,yield_and_group,as,yield,begin_expr,sum_of,div,end_expr,model,in,sumOf,with,gt,modelAs, modelAsAggregate, groupByAndYield, within, prevMonth},
  numbers=left, numbers=left, numberstyle=\tiny, basicstyle=\scriptsize, stepnumber=1, numbersep=5pt, keywordstyle=\color{dkgreen}, stringstyle=\color{blue}}

  \begin{code}{Entity Query Example}{lst:eql1}
    \begin{lstlisting}
select(Vehicle.class).
where().
  prop("vehModel.make").eq().val("MERCEDES").
  and().
  sumOf().prop("fuelUsages.qty").with().prop("fuelUsages.fuelType").eq().val("UNLEADED").
    gt().
  sumOf().prop("fuelUsages.qty").with().prop("fuelUsages.fuelType").eq().val("PREMIUM").
model();
    \end{lstlisting}
   \end{code}

  \paragraph{Relational concepts in object-oriented querying.}
  One of the key requirements for more advanced data querying is the ability to base queries on the results of other queries (i.e. subqueries in the FROM clause).
  Such support is well established in SQL, but is completely lacking in current ORM solutions.
  The challenge to develop this mechanism is in bringing a pure relational concept of working with sets of tuples into the object-oriented domain model.
  However, realising that any object can be decomposed into its properties, the notion of data tuple naturally translates into the tuple of properties from existing domain entities that preserve the knowledge of its type and metadata.
  The following example illustrates the use of this functionality.
  This query retrieves previous month maintenance cost, modelled by a separate non-persistent class PrevMonthMaintCost, for all Mercedes vehicles that exceeded the amount of 500.00.
  
  \begin{code}{Entity Query Example}{lst:eql2}
    \begin{lstlisting}
select(
    select(WorkOrder.class).
    where().prop("transDate").within().prevMonth().
    groupByAndYield().prop("vehicle").
    yield().sumOf().prop("totalCost").as("cost").
    model()  
).
where().
  prop("cost").gt().val(500.00).
  and().
  prop("vehicle.vehModel.make").eq().val("MERCEDES").
model(PrevMonthMaintCost.class);
    \end{lstlisting}
   \end{code}  
  
  
%   An example of an EQL query is provided as code listing~\ref{lst:eql}.  
%   It calculates the average yearly (over the period from 2009 to 2011) maintenance cost of vehicles manufactured by ``MERCEDES'' and their model names starting with ``315'', ``316'' or ``VITO'', for every sector of division ``NORTH''\footnote{This example is specific to a certain fleet domain and the full comprehension of the query implies good understanding of that domain.}.  
% 
%   \lstset{language=Java,morekeywords={val,select,where,prop,eq,and,begin,end,starts_with,any_of_values,values,or,yield_and_group,as,yield,begin_expr,sum_of,div,end_expr,model_as_aggregate,year_of,in},numbers=left, numberstyle=\tiny, basicstyle=\scriptsize, stepnumber=1, numbersep=5pt, keywordstyle=\color{dkgreen}, stringstyle=\color{blue}}
%   \begin{code}{EQL Query Example}{lst:eql}
%   \begin{lstlisting}
%     select(WorkOrder.class).
%     where().
%     prop("vehicle.model.make.key").eq().val("MERCEDES").and().
%     begin().
%       prop("vehicle.model.key").starts_with().any_of_values("315", "316").or().
%       prop("vehicle.model.key").eq().val("VITO").
%     end().and().
%     year_of().prop("actualStart").in().values(2009, 2010, 2011).and().
%     prop("vehicle.station.zone.sector.division.key").eq().val("NORTH").
%     yield_and_group().prop("vehicle.station.zone.sector").as("sector").
%     yield().
%       begin_expr().
% 	sum_of().prop("actualCost").div().val(3).
%       end_expr().as("averageYearlyMaintenanceCostPerSector").
%     model_as_aggregate();
%   \end{lstlisting}
%   \end{code}
%   
%   The above EQL query is interesting in many ways.
%   It shows proper support for working at the domain level referencing the nested structure of domain entities, multiple EQL constructs (aggregation, grouping, logical operations) and the capability of working with expressions\footnote{
%     Expressions in TG are implemented as an internal DSL for use by developers as well as external DSL for use directly from business applications by power users. 
%     All arithmetical operations, specially provided functions (e.g. year\_of(date)) and aggregations can be represented as type-safe expressions.
%   }
%   (summation of property actualCost divided by 3).  
  
%   A less obvious, but very powerful, feature demonstrated in this query is the ability to produce dynamically composed results that do not necessarily match any of the pre-built domain entity types.
%   This capability is critical for many real-life business requirements where some business logic or analysis may require ad-hoc data retrieval without modifying the domain itself.
%   The declarative nature of EQL significantly simplifies creation of complex queries with nested sub-queries, grouping and aggregation, which is extremely important for implementing analytical solutions.  

%   Entity Query Language fully reuses the metadata associated with domain entity types.
%   This facilitates dynamic optimisation of query execution at the databases level, which is possible due to semantic transition between the metadata of the domain and SQL.
%   For example, if some property of a domain entity, which represents a reference to another domain entity, is declared as required, then the query execution mechanism would recognise such references as SQL's \emph{INNER JOIN} instead of \emph{OUTER JOIN}, which significantly speeds up query execution.

  The EQL execution mechanism works equally well with small and large datasets by supporting partitioned data loading.
  This does not require any additional work on the part of developers -- they just need to provide the query and the system will automatically execute it with the necessary level of granularity.
  It is important that the platform does not rely on any RDBMS vendor specific features such as cursors, which might require keeping resulting datasets open in memory.
  Instead, the EQL execution mechanism automatically enhances queries to retrieve data sequentially.
  This approach naturally fits into the stateless Resources Oriented Architecture of the platform to properly support the scalability of TG-based applications.

  Developers working directly with SQL always ensure that only the required data should be returned as the result of query execution.  
  Trident Genesis supports the fine-tuning of EQL queries that match the flexibility of SQL.
  Unlike, many other existing object-relational technologies that either follow the lazy-loading model\footnote{
  This causes severe performance issues known as ``$n+1$ request'', which is especially dangerous when the technology is used by novice developers.}
  or provide ways to specify fetching as part of queries\footnote{Or even more restrictively -- as part of the object-relational mapping.}, the platform supports a concept of \emph{explicit fetching strategy}.
  This separates EQL queries, which indicate how to find data, from the fetching models, which indicate precisely what data needs to be returned, thus providing a full control over the resulting object graph initialisation.
  So, depending on business requirements the same EQL query can be executed with different fetch models, which provides a way to optimise the load time and memory footprint.

  As mentioned before, the development model hides many technical details including the data communication mechanism.
  This completely removes the conceptual mismatch between the ways of querying the domain against databases and against web resources.
  Any EQL query can be seamlessly executed as part of either client- or server-side applications.
  Thus, EQL provides a uniform high-level domain-oriented communication concept.